\documentclass[10pt,a4paper]{article}
\usepackage[utf8]{inputenc}
\usepackage[T1]{fontenc}
\usepackage[slovene]{babel}
\usepackage{lmodern}
\usepackage{enumitem}
\usepackage{amsmath}
\usepackage{amssymb}
\usepackage[margin=0.2in]{geometry}
\usepackage{flafter}
\usepackage{setspace}
\usepackage{parskip}
\usepackage{booktabs}
\usepackage{multicol}
\usepackage{pifont}% http://ctan.org/pkg/pifont
\usepackage{wrapfig,lipsum,booktabs}
\setlength{\parskip}{0pt}
\DeclareMathOperator{\Ima}{Im}
\DeclareMathOperator{\Int}{Int}
\DeclareMathOperator{\Fr}{Fr}
\DeclareMathOperator{\Cl}{Cl}

\newcommand{\TT}{\mathcal{T}}
\newcommand{\BB}{\mathcal{B}}
\newcommand{\PP}{\mathcal{P}}
\newcommand{\UU}{\mathcal{U}}
\newcommand{\CC}{\mathcal{C}}

% \newcommand{\cmark}{\ding{51}}%
\newcommand{\cmark}{\checkmark}%
\newcommand{\xmark}{\ding{55}}%


\title{STOP in UGT}
\author{}
\date{}

\begin{document}
% \linespread{}

% \section{STOP}
% \setlength{\textfloatsep}{0pt}  %% Or whatever length
% \setlength\lineskip{0pt}
\begin{enumerate} 
    \setlength\itemsep{-1px}%%% Vrednosti so lahko negativne

\item (p) Homeom.: $f: [a,b]\to[c,d]$: $f(x)=\frac{d-c}{b-a}(x-a)+c$,
$g: (-1,1)\to\mathbb{R}$: $g(x)=\frac{x}{1-|x|}$ in 
$h: \mathbb{R}\to(-1,1)$: $h(x)=\frac{x}{1+|x|}$
            
\item (D) Prostor $X$ zadošča prvemu aksiomu števnosti (je $1$-števen), če ima vsaka točka $X$ kako števno bazo okolic.
            
\item (D) Prostor zadošča drugemu aksiomu števnosti (je $2$-števen), če $\exists$ kaka števna baza za njegovo topologijo.
            
% \item (I) Metrični prostor $(X,d)$ je $2$-števen natanko takrat, ko v njem $\exists$ števna povsod gosta podmnožica.
\item (I) Metričnost $\implies$ (2-števnost $\iff$ $\exists$ števna povsod gosta podmnožica)
            
\item (D) Podmnožica $A$ je povsod gosta v $X$, če seka vsako odprto množico $X$ (tako kot $\mathbb{Q}$ v $\mathbb{R}$), ali ekvivalentno, če je $\overline{A}=X$.
            
\item (D) Prostor je separabilen, če premore kako povsod gosto podmnožico, ki je števna.
            
\item (I) Če je prostor $2$-števen, je tudi separabilen in $1$-števen (obrat ne velja).
            
\item (O) V metričnih prostorih je $2$-števnost ekvivalentna separabilnosti.
            
\item (I)  prostor $Y$ Hausdorffov.~(1) Vsaka končna podmnožica $Y$ je zaprta.~Točke so zaprte.~(2) Točka $y$ je stekališče množice $A\subseteq Y$ natanko takrat, ko vsaka okolica $y$ vsebuje neskončno točk iz $A$.~(3) Zaporedje v $Y$ ima največ eno limito.~(4) Množica točk ujemanja $\left\{x\in X | f(x)=g(x)\right\}$ je zaprta v $X$ za poljubni preslikavi $f,g: X\to Y$.~(5) Če se preslikavi $f,g: X\to Y$ ujemata na neki gosti podmnožici $X$, potem je $f=g$.~(6) Graf preslikave $f: X\to Y$ je azprt podprostor produkta $X\times Y$.
            
\item (D) Aksiomi ločljivosti: Naj bo $(X,\TT)$ topološki prostor. $X$ je: 
\item 
% \vspace{0px}
            \begin{itemize}
                \setlength\itemsep{-4px}
                \item $T_0$: Za različni točki $x,x'\in X$ obstaja okolica ene izmed točki $x,x'$, ki jo loči od druge točke.
                \item $T_1$: Za različni točki $x,x'\in X$ obstaja okolica točke $x$, ki jo loči od $x'$ in obenem obstaja okolica točke $x'$, ki jo loči od $x$.
                \item $T_2$: Za različni točki $x,x'\in X$ obstajata okolici, ki ostro ločita $x$ in $x'$.
                \item $T_3$: Za točki $x\in X$ in zaprto množico $A\subseteq X$, ki ne vsebuje $x$, obstajata okolici, ki ostro ločita $x$ in $A$.
                %\item $T_{3\frac{1}{2}}$: $a\in X$ in $A$ zaprta množica, ki ne vsebuje $a$. Če za vsak par $(a,A)$ obstaja zvezna funkcija $\varphi:X\to[0,1]$, da velja $\varphi(a)=1$ in $\varphi(x)=0,\forall x\in A$.
                \item $T_4$: Za disjunktni zaprti množici $A,B\subseteq X$ obstajata okolici, ki ostro ločita $A$ in $B$.
            \end{itemize}
            
\item (O) Fréchetova lastnost: $T_1$; Hausdorffova lastnost: $T_2$; regularnost: $T_0$ in $T_3$; normalnost: $T_1$ in $T_4$.
            
\item (T) Prostor $X$ ima lastnost $T_3$ natanko takrat, ko za vsak $x\in X$ in vsako odprto okolico $U$ za $x$ $\exists$ taka odprta množica $V$, da velja $x\in V\subseteq \overline{V}\subseteq U$.
            
\item (I) Prostor, ki je regularen in $2$-števen, je normalen.
            
\item 
\begin{itemize}[label={}]
    \setlength\itemsep{-4px}
    \item \(\iff\) Prostor $X$ je nepovezan
    \item \(\iff\) Prostor $X$ je disjunktna unija dveh nepraznih zaprtih podmnožic
    \item \(\iff\) V prostoru $X$ $\exists$ prava neprazna pomnožica, ki je hkrati odprta in zaprta
    \item \(\iff\) $\exists$ surjektivna preslikava $f:X\to\{0,1\}$
\end{itemize}
            
\item (I)  $X$ kompakten, $Y$ pa Hausdorffov prostor.~(1) Vsaka preslikava $f:X\to Y$ je zaprta.~(2) Vsaka injektivna preslikava $f:X\to Y$ je vložitev.~(3) Vsaka bijektivna preslikava $f:X\to Y$ je homeomorfizem.
            
\item (I) (Urisonova lema) Hausdorffov prostor $X$ je normalen natanko takrat, ko za poljubni disjunktni neprazni zaprti podmnožici $A,B\subseteq X$ $\exists$ preslikava $f:(X,A,B)\to(I,0,1)$.
    
\item \(X\) lokalno povezan \(\iff\) komponente vsake odprte okolice so odprte v \(X\)
\item Komponente za povezanost so zaprte, nenujno odprte
% \item (I) (Tietzejev razširitveni izrek)  $A$ zaprt podprostor normalnega prostora $X$ in $J\subseteq\mathbb{R}$ poljuben interval.~Tedaj vsako preslikavo $f:A\to J$ lahko razširimo do preslikave $F:X\to J$.
        \end{enumerate}

\vspace{-4mm}
\begin{table}[htbp]
    \begin{tabular}{|c|c|c|c|c|}\hline
        Lastnost                & dedna & prod. & delj. & se ohranja pri    \\ \hline
        $T_1$ (Fréchet)         & \cmark& \cmark& \xmark&                   \\ \hline
        $T_2$ (Hausdorff)       & \cmark& \cmark& \xmark& zaprte bijekcije  \\ \hline
        $T_3$                   & \cmark& \cmark& \xmark&                   \\ \hline
        $T_4$                   &       &   	& \xmark&                   \\ \hline
        $1$-števnost            & \cmark& števno& \xmark&                   \\ \hline
        $2$-števnost            & \cmark& števno& \xmark& odp. zvezne sur.  \\ \hline
        separabilnost           & odprto& števno& \cmark& zvezna            \\ \hline
        regularnost ($T_3, T_0$)& \cmark& \cmark&       &                   \\ \hline
        normalnost ($T_4, T_1$) & zaprto&       &       & zap. zvezne sur.  \\ \hline
        absolutni ekstenzor     &       & \cmark&       &                   \\ \hline
    \end{tabular}
    \begin{tabular}{|c|c|c|c|c|} \hline
        Lastnost          & dedna & prod  & delj. & se ohranja pri\\ \hline
        diskretnost       & \cmark& končno& \cmark&               \\ \hline
        metrizabilnost    & \cmark& števno& \xmark&               \\ \hline
        kompaktnost       & zaprto& \cmark& \cmark& zvezne        \\ \hline
        lok. kompaktnost  &       & \cmark& \xmark& odp. zvezne   \\ \hline
        povezanost        &       & \cmark& \cmark& zvezne        \\ \hline
        lok. povezanost   &       & \cmark& \cmark& odp./zap. zv.\\ \hline
        pov. s potmi      &       & \cmark& \cmark& zvezne        \\ \hline
        lok. pov. s potmi &       & končno& \cmark& odp./zap. zv. \\ \hline
        popolna nepov.    &       & končno& \xmark&               \\ \hline
    \end{tabular}
\end{table}
\vspace{-4mm}


% \begin{table}[htbp]
%     % \centering
%     \begin{tabular}{|c|c|}
%     \hline

%     Lastnost &  \\ \hline \hline
%     $T_1$ (Fréchet) & dedna, produktna, nedeljiva \\ \hline
%     $T_2$ (Hausdorff) & dedna, produktna, zaprte bijekcije, nedeljiva \\ \hline
%     $T_3$ & dedna, produktna, nedeljiva \\ \hline
%     $T_4$ & nedeljiva \\ \hline
%     $1$-števnost & dedna, števno-produktna, nedeljiva \\ \hline
%     $2$-števnost & dedna, števno-produktna, odprte zvezne surjekcije, nedeljiva \\ \hline
%     separabilnost & odprto-dedna, zvezna, števno-produktna, deljiva \\ \hline
%     regularnost ($T_3, T_0$) & dedna, produktna \\ \hline
%     normalnost ($T_4, T_1$) & zaprto-dedna, zaprte zvezne surjekcije \\ \hline
%     % \end{tabular}
%     % \begin{tabular}{|c|c|}
%     diskretnost & dedna, končno-produktna, deljiva \\ \hline
%     metrizabilnost & dedna, števno-produktna, nedeljiva \\ \hline
%     kompaktnost & produktna, zvezne, zaprto-dedna, deljiva \\ \hline
%     lokalna kompaktnost & odprte zvezne, produktna(kompaktnost), nedeljiva \\ \hline
%     povezanost & produktna, zvezne, deljiva \\ \hline
%     lokalna povezanost & produktna(povezanost), odprte zvezne, zaprte zvezne, deljiva\\ \hline
%     povezanost s potmi & produktna, zvezne, deljiva \\ \hline
%     lokalna povezanost s potmi & končno-produktna, odprte zvezne, zaprte zvezne, deljiva \\ \hline
%     popolna nepovezanost & končno-produktna, nedeljiva \\ \hline
%     \end{tabular}
%     % \caption{Pregled topoloških lastnosti}
%     % \label{tab:top_properties}
%     \end{table}


% \begin{table}[htbp]

%     \tiny\centering
%     \begin{tabular}{|c|c|c|c|}
%     \hline

%     Lastnost &  & Lastnost & \\ \hline \hline
%     $T_1$ (Fréchet) & dedna, produktna & diskretnost & dedna, končno-produktna, kvocientna \\ \hline
%     $T_2$ (Hausdorff) & dedna, produktna, zaprte bijekcije, nekvocientna & metrizabilnost & dedna, števno-produktna, nekvocientna \\ \hline
%     $T_3$ & dedna, produktna & kompaktnost & produktna, zvezne, zaprto-dedna, kvocientna \\ \hline
%     $T_4$ &  & lokalna kompaktnost & odprte zvezne, produktna(kompaktnost), nekvocientna \\ \hline
%     $1$-števnost & dedna, števno-produktna, nekvocientna & povezanost & produktna, zvezne, kvocientna \\ \hline
%     $2$-števnost & dedna, števno-produktna, odprte zvezne surjekcije, nekvocientna & lokalna povezanost & produktna(povezanost), odprte zvezne, zaprte zvezne, kvocientna \\ \hline
%     separabilnost & odprto-dedna, zvezna, števno-produktna, kvocientna & povezanost s potmi & produktna, zvezne, kvocientna \\ \hline
%     regularnost ($T_3, T_0$) & dedna, produktna & lokalna povezanost s potmi & končno-produktna, odprte zvezne, zaprte zvezne, kvocientna \\ \hline
%     normalnost ($T_4, T_1$) & zaprto-dedna, zaprte zvezne surjekcije & popolna nepovezanost & končno-produktna, nekvocientna \\ \hline
%     \end{tabular}
%     \label{tab:top_properties}
%     \end{table}

% \section{UGT}
% \subsection{Prvi kolokvij}
% \noindent\makebox[\linewidth]{\rule{\paperwidth}{0.4pt}}
\begin{enumerate}
    % \setlength\itemsep{2px}%%% Vrednosti so lahko negativne
    
\item \textbf{Kvocientna množica:}  \(X\) topološki prostor in \(\sim\)
    ekvivalenčna relacija na \(X\).~\textit{Kvocientna množica} \(X/{\sim}\) je množica
    vseh ekvivalenčnih razredov v \(X\) glede na \(\sim\), opremljena z najmočnejšo
    topologijo, tj.~topologijo \(\tau\) na \(X/{\sim}\), za katero velja:
    $ V \in \tau \iff p^{-1}(V) \subseteq X \text{ je odprta za vsako } V
    \subseteq X/{\sim}, $ kjer je \(p: X \rightarrow X/{\sim}\) kvocientna projekcija.
    
\item \textbf{Kvocientna projekcija:}  \(X\) topološki prostor in \(\sim\)
    ekvivalenčna relacija na \(X\).~\textit{Kvocientna projekcija}
    \(q: X \rightarrow X/{\sim}\) je preslikava, ki vsakemu elementu \(x\) priredi
    njegov ekvivalenčni razred \([x]\).
    
\item Za podmnožico \(A\) prostora \(X\) je kvocientna projekcija \(q(A)\)
    odprta/zaprta v kvocientnem prostoru \(X/{\sim}\) natanko tedaj, ko je nasičenje
    \(A\), označeno z \(q^{-1}(q(A))\), odprto/zaprto v originalnem prostoru \(X\).
    
\item \textbf{Inducirana preslikava:}  \(f: X \rightarrow Y\) preslikava med
    topološkima prostoroma \(X\) in \(Y\), ter \(\sim\) ekvivalenčna relacija na \(X\).
    Če je preslikava $f$ konstantna na ekvivalenčnih razredih, je njena \textit{inducirana preslikava} 
		\(f_{\sim}: X/{\sim} \rightarrow Y\),
    definirana kot \(f_{\sim}([x]) = f(x)\) za vsak \(x \in X\), kjer je \([x]\)
    ekvivalenčni razred elementa \(x\) glede na \(\sim\).
    
\item $f = \bar{f} \circ q $, preverimo: $f$ dela prave identifikacije:
    $x \sim y \Leftrightarrow f(x) = f(y)$, zvezna, surjektivna,
    kvocientna v ožjem smislu.
    
\item Surjektivna preslikava $f: X \rightarrow Y$ je kvocientna v ožjem smislu
    $\Leftrightarrow$ slika nasičene odprte podmnožice $X$ v odprte podmnožice $Y$
    $\Leftrightarrow$ slika nasičene zaprte podmnožice $X$ v zaprte podmnožice $Y$.
    
\item  $r: X \rightarrow Y$ in $s: Y \rightarrow X$ zvetni, za
    kateri velja $r \circ  s = Id_Y$ (``$r$ je retrakcija, $s$ je prerez'').~Tedaj je $r$
    kvocientna in $s$ vložitev.
    
\item  $f : X \rightarrow Y$ zvezna preslikava in $(K_i \subseteq  X)_{i \in I}$ pokritje
    $X$ s kompaktnimi podprostori.~Privzemimo, da je družina $(f_*(K_i))_{i \in I}$
    lokalno končna v $Y$ in da je $Y$ Hausdorffov.~Tedaj je f zaprta.~Posledično,
    če je $f$ dodatno še surjektivna, je kvocientna.
    
\item  $f : X \rightarrow Y$ zvezna preslikava, $Y$ Hausdorffov in $K \subseteq X$
    kompakt, za katerega velja $f_*(K) = Y$.~Tedaj je $f$ kvocientna.
    
\item \textbf{Konveksna kombinacija}: daljico $AB$ slikamo v daljico $CD: $
    $(1-t)A + tB \mapsto (1-t)C + tD$.
    
\item $I := [0, 1], X = I^2, (0, y) \sim (1, y), (x, 0) \sim (x, 1).~X/\sim \
    \approx S^1 \times S^1$.~Preslikava: $f: I \times I \rightarrow S^1 \times S^1,
    (x, y) \mapsto (e^{2 \pi i x}, e^{2 \pi i y})$
    
\item Model \textbf{Möbiusovega traku} znotraj polnega torusa: $f: [0, 1] \times [-1, 1]
    \rightarrow S^1 \times B^2, f(x, y) = (e^{2 \pi i x}, e^{\pi i x}y)$
    
% \item \textbf{Deljive} top.~lastnosti: kompaktnost, povezanost (s potmi), separabilnost,
    % lok.~povezanost (s potmi), diskretnost, trivialnost
    
% \item \textbf{Nedeljive} top.~lastnosti: ločljivostne, lok.~kompaktnost, 1- in 2-števnost,
    % metrizabilnost, popolna nepovezanost.
    
\item \textbf{Top grupa} $G$ je grupa in top.~pr., množenje $m: G \times
    G \rightarrow G, (a, b) \mapsto ab$ in invertiranje $inv: G \rightarrow G, a
    \mapsto a^{-1}$ sta zvezni.
    
\item \textbf{Leva translacija} $L_a : G \rightarrow G, g \mapsto ag$ in
    \textbf{desna translacija} sta homeomorfizma.
    
\item Topološka grupa je \textbf{homogen prostor}, tj.~$\forall a, b \in G \
    \exists$ homeo $h: G \rightarrow G, a \mapsto b$.
    
\item \textbf{Delovanje} $G$ na $X$ je zv.~preslikava $\varphi: G \times X
    \rightarrow X, (g, x) \mapsto g \cdot x$, za katero velja $\varphi(e, x) =
    e \cdot x = x, h \cdot (g \cdot x) = (hg) \cdot x$.
    
\item Ekv.~relacija glede na delovanje: $x \sim y \Leftrightarrow \exists g \in G:
    y = g \cdot x$
    \textbf{Stabilizator} elementa $x$ je $G_x = \{g \in G | g \cdot x = x\}$

    
\item \textbf{Orbita} elementa $x$ je $G \cdot x = \{g \cdot x | g \in G\} = [x]$, \textbf{Prostor orbit} je kvoc.~prostor $X/G$ pri tej relaciji.
    
% \item \textbf{Stabilizator} elementa $x$ je $G_x = \{g \in G | g \cdot x = x\}$
    
\item Če top.~grupa $G$ deluje na top.~prostoru $X$, potem je kvoc.~projekcija
    $q: X \rightarrow X/G$ odprta.
    
\item Za top.~grupo $G$ velja: (a) Množica $A \subseteq G$ je okolica točke
    $a \in G$ natanko tedaj, ko je $ba^{-1}A$ okolica točke $b \in G$.
    (b) Če je $H \leq G$ podgrupa in okolica enote v $G$, je $H$ odprta in zaprta v $G$.
    (c) Če je C komponenta za povezanost, ki vsebuje enoto $G$, je $C$ zaprta edinka v $G$.
    (d) Za $G$ so lastnosti $T_0, T_1, T_2$ ekvivalentne.
    
\item \textbf{Stožec} nad $X$ je $CX = X \times [0, 1]/X \times \{1\}$, \textbf{Suspenzija} nad $X$ je $\Sigma  X = X \times [-1, 1]/
    \{X \times \{-1\}, X \times \{1\}\}$.
    
\item \textbf{Zlepek}: $X, Y$ top.~prostora, $A \subseteq X, f: A \rightarrow Y$
    zvezna.~$X \cup_f Y = (X + Y)/a \sim f(a) \forall a \in A$.
    
\item \textbf{Normalnost zlepkov}: $X, Y$ normalna, $A^{\text{zap}} \subseteq X, f:
    A \rightarrow Y$ zv.~$\Rightarrow$ zlepek $X \cup_f Y$ normalen.
    
\item $A^{\text{zap}} \subseteq X, f: A \rightarrow Y$ zap vložitev.~$Z = X \cup_f Y$
    $\Rightarrow$ (1) $X, Y$ 2-števna $\Rightarrow Z$ 2-števen,
    (2) $X, Y$ regularna $\Rightarrow Z$ regularen.
    
\item $q \circ in_1: X \rightarrow X \cup_f Y$ v splošnem ni vložitev, ampak $q \circ in_2: Y \rightarrow X \cup_f Y$ je vložitev.
    
\item Injekcije so zvezne, odprte in zaprte vložitve v vsoto.
    
\item Dogovor: za $X / \emptyset$ dobimo $X$ s še eno točko, tj.~podprostor $A$
    se v kvocientu vedno nadomesti z eno točko.

\vspace{-3mm}
\noindent\makebox[\linewidth]{\rule{\paperwidth}{0.4pt}} 
\vspace{-5mm}

\end{enumerate}
%%%%%%%%%%%%%%%%%%%%%%%%%%%%%%%%%%%%%%%%%%%%%%%%%%%%%%%%%%%%%%%
%%%%%%%%%%%%%%%%%%%%%%%%%%%%%%%%%%%%%%%%%%%%%%%%%%%%%%%%%%%%%%%
%%%%%%%%%%%%%%%%%%%%%%%%%%%%%%%%%%%%%%%%%%%%%%%%%%%%%%%%%%%%%%%
%%%%%%%%%%%%%%%%%%%%%%%%%%%%%%%%%%%%%%%%%%%%%%%%%%%%%%%%%%%%%%%
%%%%%%%%%%%%%%%%%%%%%%%%%%%%%%%%%%%%%%%%%%%%%%%%%%%%%%%%%%%%%%%
%%%%%%%%%%%%%%%%%%%%%%%%%%%%%%%%%%%%%%%%%%%%%%%%%%%%%%%%%%%%%%%
%%%%%%%%%%%%%%%%%%%%%%%%%%%%%%%%%%%%%%%%%%%%%%%%%%%%%%%%%%%%%%%
%%%%%%%%%%%%%%%%%%%%%%%%%%%%%%%%%%%%%%%%%%%%%%%%%%%%%%%%%%%%%%%
%%%%%%%%%%%%%%%%%%%%%%%%%%%%%%%%%%%%%%%%%%%%%%%%%%%%%%%%%%%%%%%
%%%%%%%%%%%%%%%%%%%%%%%%%%%%%%%%%%%%%%%%%%%%%%%%%%%%%%%%%%%%%%%
%%%%%%%%%%%%%%%%%%%%%%%%%%%%%%%%%%%%%%%%%%%%%%%%%%%%%%%%%%%%%%%
% \subsection{Drugi kolokvij}
% \fontsize{12pt}{0px}
\begin{enumerate}
    \setlength\itemsep{3px}%%% Vrednosti so lahko negativne
    
% \item \textbf{Retrakcija}: $A \subseteq X, r: X \rightarrow A$ zvezna, je retrakcija, če $r|_A = id_A$, $A$ je retrakt $X$, če $\exists$ retrakcija.

\vspace{-2mm}
\begin{table}[htbp]
    \begin{tabular}{p{14cm}}
        $A \subseteq X, r: X \rightarrow A$ zvezna, je \textbf{retrakcija}, če $r|_A = id_A$, $A$ je \textbf{retrakt} $X$, če $\exists$ retrakcija \\
    \end{tabular}
    \begin{tabular}{|l|}
        \hline
        če $X \in T_2$, je retrakt zaprt \\
        \hline
    \end{tabular}
\end{table}
\vspace{-4.5mm}

\item (T)Retrakt povezanega/kompaktnega prostora je povezan/kompakten, retrakt absolutnega ekstenzorja je absolutni ekstenzor

% \vspace{-3mm}
% \noindent\makebox[\linewidth]{\rule{\paperwidth}{0.4pt}} 
% \vspace{-3mm}

\vspace{-3mm}
\begin{table}[htbp]
    \begin{tabular}{p{11.8cm}}
        (D) \(X\) ima \textbf{LNT}, če ima \(\forall\) zvezna preslikava \(X \rightarrow X\) negibno točko \\ 
    \end{tabular}    
    \begin{tabular}{|l|}
        \hline
        $X$ ima LNT, $A$ retrakt $X$ $\Rightarrow$ $A$ ima LNT \\
        \hline
    \end{tabular}
\end{table}
\vspace{-4mm}

% \vspace{-3mm}
% \noindent\makebox[\linewidth]{\rule{\paperwidth}{0.4pt}} 
% \vspace{-5mm}

\item (D) $\mathcal{R}$ razed top.~pr., $Y$ top.~pr., $Y \in AE(\mathcal{R})$, če za $\forall X \in \mathcal{R}, \forall A^{zap} \subseteq X, \forall f: A \rightarrow Y$ zv.\ $\exists F: X \rightarrow Y$ zv.: $F|_A = f$ ($F$~razširitev~$f$)

\vspace{-4mm}
\begin{table}[htbp]
    \centering
    \begin{tabular}{|l|}
        \hline
        \(\forall\) interval $\subseteq \mathbb{R}$ je $AE(\mathcal{N})$ \\
        $\mathcal{R} \cap AE(\mathcal{R}) \subseteq AR(\mathcal{R})$ \\
        \hline
    \end{tabular}
    \begin{tabular}{l}
        Retrakt povezanega/kompaktnega prostora je povezan/kompakten \\ 
        retrakt absolutnega ekstenzorja je absolutni ekstenzor \\
    \end{tabular}
    \begin{tabular}{|c|}
        \hline
        Če je X kontraktibilen \\
        je povezan s potmi \\
        \hline
    \end{tabular}
\end{table}
\vspace{-4mm}

% \vspace{-3mm}
% \noindent\makebox[\linewidth]{\rule{\paperwidth}{0.4pt}} 
% \vspace{-5mm}
    
\item (D) $f,g: X \rightarrow Y$ zv., \textbf{homotopija} od $f$ do $g$ je zv.~preslikava $H: X \times I \rightarrow Y$ z $H(x, 0) = f(x)$ in $H(x, 1) = g(x)$
    
\item (D) Prostor X je \textbf{kontraktibilen}, če je $id_X$ homotopna ($\simeq$) kakšni konst.~preslikavi $X \rightarrow X$; homotopija = kontrakcija
    
% \item Če je X kontraktibilen, je povezan s potmi

\vspace{1mm}
% \begin{table}[htbp]
    \begin{tabular}{|l|}
        \hline
        $A_n:\forall f : \mathbb{B}^n \rightarrow \mathbb{B}^n$ zvezna ima negibno točko \\
        $C_n: S^{n-1}$ ni kontraktibilna \\
        $B_n: S^{n-1}$ ni retrakt $B^n$ \\
        $D_n: D \subseteq \mathbb{R}^n$  topološki $k$-disk za $0 < k \leq n$, torej $D \approx B^k$. $D$ ne deli $\mathbb{R}^n$ \\    
        \hline
    \end{tabular}
% \end{table}
% \vspace{-4mm}


% \vspace{-3mm}
% \noindent\makebox[\linewidth]{\rule{\paperwidth}{0.4pt}} 
% \vspace{-5mm}

\item (I)(Jordan-Brouwer)  $n \geq 2$, $S \subseteq \mathbb{R}^n$ top.~ ($n-1$)-sfera, ($S \approx S^{n-1}$). Tedaj ima $\mathbb{R}^n \setminus S$ dve komponenti, eno omejeno in eno neomejeno, obe sta odprti v $\mathbb{R}^n$ in povezani s potmi, $S$ je meja obeh.
        
\item (I)(Schönfliesov) $S^1 \approx S \subseteq \mathbb{R}^2$, $V$ omejena komponenta $\mathbb{R}^2 \setminus S \Rightarrow Cl(V) \approx B^2$

% \vspace{-3mm}
% \noindent\makebox[\linewidth]{\rule{\paperwidth}{0.4pt}} 
% \vspace{-5mm}

\item (D) $S \approx S^{n-1} \subseteq \mathbb{R}^n$ je \textbf{lokalno ploščata} v $x \in S$, če $\exists V: x \in V^{odp} \subseteq \mathbb{R}^n$ in $\exists$ homeo $h: V \rightarrow W$, $W^{odp} \subseteq \mathbb{R}^n$,
    da je $h_\ast(S \cap V) = (\mathbb{R}^{n-1} \times \{ 0\}) \cap W$. \quad $S$ je \textbf{lokalno ploščata}, če je taka v vseh svojih točkah (vložitev je krotka ali $S$ je podmnt);
    če v $x \in S, S$ ni lokalno ploščata, je $x$ \textbf{divja točka}
 
% \vspace{-3mm}
% \noindent\makebox[\linewidth]{\rule{\paperwidth}{0.4pt}} 
% \vspace{-5mm}

\item (I)(Brouwer)(o odprti preslikavi) $U^{odp} \subseteq \mathbb{R}^n, f:U \rightarrow \mathbb{R}^n$ zv., inj.~$\Rightarrow f$ odp.~preslikava ($\Rightarrow f_\ast(U)^{odp} \subseteq \mathbb{R}^n$)
    
\item (invarianca odp mn) $U, V \subseteq \mathbb{R}^n, U^{odp}, U \approx V \Rightarrow V^{odp} \subseteq \mathbb{R}^n$
    
\item Posledica: $A,B \subseteq \mathbb{R}^n, A \approx B \Rightarrow \Int(A) \approx \Int(B)$ in $\partial(A) \approx \partial(B)$

% \vspace{-3mm}
% \hrulefill
% \vspace{-1.5mm}

\item (D) topološka \textbf{mnogoterost} dimenzije~$n \in \mathbb{N}$ ($n$-mnt) je $T_2$, 2-števen topološki prostor $M$, v katerem ima vsaka točka odprto okolico homeo $\mathbb{R}^n$ ali $\mathbb{R}^{n}_+$.
    Okolico $V$ imenujemo \textbf{evklidska okolica}, 
    home. ($h: V \stackrel{\approx}{\rightarrow} \mathbb{R}^n$ ali $\mathbb{R}^{n}_+$) imenujemo \textbf{karta} na $M$.

\item komponenta $n$-mnt je $n$-mnt (komponente so odprte, ker je mnt lok pov $\Leftarrow$ lok evklidskost) 
\item disj unija (največ) števno mnogo $n$-mnt je $n$-mnt 
\item Rob mnt: $M$ $n$-mnt, $\Int(M)^{odp} \subseteq M \Rightarrow \partial M^{zap} \subseteq M$

% \vspace{-3mm}
% \hrulefill
% \vspace{-1.5mm}

\item (I)(o odprti preslikavi za mnt) $M$, $N$ $n$-mnt, $V^{odp} \subseteq int(M), f: V \rightarrow N$ zv., inj.~$\Rightarrow f$ odp in $f_\ast (V) \subseteq int(N)$
\item če je $M$ kompaktna $\Rightarrow \partial M$ sklenjena (op: rob kompaktne ploskve je disjunktna unija krožnic) 

% \vspace{-4mm}
% \begin{table}[htbp]
    \begin{tabular}{l}
        (I) $M$ $n$-mnt, $\partial M \neq \emptyset \Rightarrow \partial M$ $(n-1)$-mnt s praznim robom $(\partial (\partial M) = \emptyset)$  \\
        ( ) $\partial M$ ne deli $M$ ($M$ povezana $\Rightarrow M \setminus \partial M$ povezana) \\
        (I) $M$ $m$-mnt, $N$ $n$-mnt $\Rightarrow M \times N$ je ($m+n$)-mnt \\
    \end{tabular}
    \begin{tabular}{|l|}
        \hline
        $\Int(M \times N) = \Int(M) \times \Int(N)$ \\
        $\partial (M \times N) = \partial M \times N \cup M \times \partial N$ \\
        \hline
    \end{tabular}
% \end{table}
% \vspace{-4mm}


% \vspace{-3mm}
% \hrulefill
% \vspace{-1.5mm}

\item Zlepek mnt: (D) $N$ $n$-mnt, $L$ $l$-mnt; $L \cap \partial N = \partial L \Rightarrow L$ \textbf{prav vložena} 
\item $L$ \textbf{lokalno ploščata} ali \textbf{podmnt}, če je prav vložena in $\forall x \in L \exists$ evklidska okolica
    $V: x \in V \subseteq N$ in homeo $h: V \rightarrow \mathbb{R}^n~\text{ali}~\mathbb{R}^{n}_+$, da je $h_\ast(L\cap V) = h_\ast(V) \cap (\{0\}^{n-l}\times \mathbb{R}^l)$
%Povej to na lepši način
\item (I) $N_1, N_2$ $n$-mnt, $L_i \subseteq \partial N_i$ ($n-1$)-mnt in $\partial L_i$ lokalno ploščat v $\partial N_i$ 
    \quad $h: L_1 \rightarrow L_2$ homeo $\Rightarrow$ zlepek $N_1 \cup_h N_2 = N$
    je n-mnt z robom, ki je zlepek $(\partial N_1 \setminus \Int(L_1)) \cup_{h|_{\partial L_1}} (\partial N_2 \setminus \Int(L_2))$ in $(N_1 \setminus L_1) \cup (N_2 \setminus L_2)$ je vložena v zlepek kot odprta podmn.
\item $N$ $n$-mnt, $L \subseteq N$ lokalno ploščata ($n-1$)-podmnt $\Rightarrow$ lahko $N$ prerežemo vzdolž $L$ 
\item (D) $M,N$ n-mnt, $D \subseteq \Int(M), E \subseteq \Int(N)$
topološka n-diska z lok ploščatima robovoma, izberimo homeo $h: \partial D \rightarrow \partial E$.\textbf{Povezana vsota} M in N je $M \# N = (M \setminus \Int(D)) \cup_h (N \setminus \Int(E))$ op: $M \# N$ je n-mnt
   
\item (I)(homogenost mnt) $M$ povezana $n$-mnt, $x, y \in int(M) \Rightarrow \exists$ homeo $h: M \rightarrow M, h(x)=y$

\vspace{-2mm}
\noindent\makebox[\linewidth]{\rule{\paperwidth}{0.4pt}} 
\vspace{-6mm}

\item Eulerjeva karakteristika in orientabilnost sta topološki lastnosti. (Karakterizacija: orientabilnost, 
$\chi$, 
št.~robnih komponent )
    
\item \textbf{Eulerjeva karakteristika}~$X$: $\chi(X) = \text{\#($2$-celic)} - \text{\#($1$-celic)} + \text{\#($0$-celic)}$


\vspace{1mm}
% \begin{table}
    \begin{tabular}{l}
        $A, B \subseteq\mathbb{R}^n$ zap, $A, B \in AE(\mathcal{N})$: \\ \hline
        $|A \cap B| = 1 \Rightarrow A \cup B \in AE(\mathcal{N})$ \\
        $A \cap B \in AE(\mathcal{N})\Rightarrow A \cup B \in AE(\mathcal{N})$ 
        
    \end{tabular}
    \begin{tabular}{|l|}
        \hline
        $\chi(nT) = 2 - 2n$ \\
        $\chi(nP) = 2 - n$ \\
        $\chi(S^2) = \chi(0T) = 2$ \\
        \hline
    \end{tabular}
    \begin{tabular}{|l|}
        \hline
        $\chi(M) = 0$ \\
        $\chi(A \# B) = \chi(A) + \chi(B) - 2$ \\
        $\chi(X \cup trak) = \chi(X) - 1$ \\
        \hline
    \end{tabular}
    \begin{tabular}{l}
        $nT$ so orientabilni \\
        $nP$ niso orientabilni \\
        $K \approx 2P$
    \end{tabular}
% \end{table}
% \vspace{-4mm}

\item $X$ kompaktna mnt: $\partial X \neq \emptyset, i: \partial X \rightarrow X$ inkluzija $\Rightarrow$ zlepek $X \cup_{\partial X} X := X \cup _i X$ kompaktna mnt brez roba (podvojitev mnogoterosti X)



\end{enumerate}

\end{document}